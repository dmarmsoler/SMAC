\RequirePackage{ifvtex}
\documentclass[10pt,DIV17,a4paper,abstract=true,twoside=semi,openright]
{scrreprt}
\usepackage[T1]{fontenc}
\usepackage[utf8]{inputenc}
\usepackage{textcomp}
\usepackage[english]{babel}
\RequirePackage[caption]{subfig}
\usepackage{isabelle}
\usepackage{isabellesym}
\usepackage{amsmath}
\usepackage{amssymb}
\usepackage{graphicx}
\usepackage{hyperref}
\usepackage{orcidlink}
\setcounter{tocdepth}{3} 
\hypersetup{%
   bookmarksdepth=3
  ,pdfpagelabels
  ,pageanchor=true
  ,bookmarksnumbered
  ,plainpages=false
} % more detailed digital TOC (aka bookmarks)
\sloppy
\allowdisplaybreaks[4]
\urlstyle{rm}
\isabellestyle{it}

% for uniform font size
%\renewcommand{\isastyle}{\isastyleminor}

\newenvironment{frontmatter}{}{}

\pagestyle{headings}
\isabellestyle{default}
\setcounter{tocdepth}{1}
\newcommand{\ie}{i.\,e.\xspace}
\newcommand{\eg}{e.\,g.\xspace}
\newcommand{\thy}{\isabellecontext}
\renewcommand{\isamarkupsection}[1]{%
  \begingroup% 
  \def\isacharunderscore{\textunderscore}%
  \section{#1 (\thy)}%
  \endgroup% 
}

\title{Isabelle/Solidity meets SMAC}
\author{Diego Marmsoler\textsuperscript{\orcidlink{0000-0003-2859-7673}}
        and Asad Ahmed}%
\publishers{
Department of Computer Science, University of Exeter, Exeter, UK\texorpdfstring{\\}{, }
\texttt{\{d.marmsoler, a.ahmed6\}@exeter.ac.uk}
}
\begin{document}
\begin{frontmatter}
\maketitle
\begin{abstract}
  \begin{quote}
Smart contracts are programs that execute on the blockchain
to automate financial transactions. As with any software, they are sus-
ceptible to bugs, which can be exploited and lead to significant economic
losses. To mitigate such risks, formal verification is increasingly employed
alongside traditional auditing methods. Solidity is the most widely used
language for developing smart contracts. A key feature of Solidity is its
support for arrays, particularly memory arrays, which are implemented
as pointer structures. These structures pose challenges for formal ver-
ification, and existing approaches for the verification of Solidity smart
contracts are typically limited to handling only simple types of arrays.
To address this limitation, we introduce a formal calculus designed to
reason about Solidity-style memory arrays. We mechanically verified the
soundness of the calculus using Isabelle. To assess its completeness, we
developed a benchmark suite for memory array verification and compared
our approach against existing state-of-the-art tools. The proposed calcu-
lus can be used to verify programs using Solidity-style memory arrays.
To demonstrate this, we integrated the calculus into an existing Solidity
verification framework and used it to verify a real-world smart contract.
	\bigskip
	\noindent\textbf{Keywords:} {Program Verification, Theorem Proving, Solidity, Isabelle} 
  \end{quote}
\end{abstract}

\tableofcontents
\cleardoublepage
\end{frontmatter}


\chapter{Introduction}

The rest of this document is automatically generated from the formalization in
Isabelle/HOL, i.e., all content is checked by Isabelle. The structure follows
the theory dependencies (see \autoref{fig:session-graph}).

\begin{figure}
  \centering
  \includegraphics[height=.9\textheight, width=\textwidth,keepaspectratio]{session_graph}
  \caption{The Dependency Graph of the Isabelle Theories.\label{fig:session-graph}}
\end{figure}

\clearpage

\chapter{Preliminaries}
This chapter contains a copy of Isabelle/Solidity.

\input{Utils}
\input{State_Monad.tex}
\input{State.tex}
\input{Solidity.tex}
\input{Contract.tex}
\input{WP.tex}

\chapter{Memory Model}

In this chapter, we present the memory model presented in the paper as well as its integration into Isabelle/Solidity.

\input{Memory.tex}
\input{Stores.tex}

\chapter{Memory Calculus}

In this chapter, we present SMAC, a calculus to reason about Solidity-style memory arrays.

\input{Mcalc.tex}

\chapter{Applications}

In this chapter we present the running example as well as the ArrayBuilder case study.

\input{Aliasing.tex}
\input{ArrayBuilder.tex}

%\IfFileExists{root.bib}{%
%    \bibliography{root}
%}{}
\end{document}

%%% Local Variables:
%%% mode: latex
%%% TeX-master: t
%%% End: